The concept of hustle culture, defined by an emphasis on constant productivity and success through overwork, has been widely discussed in both academic and popular circles. The implications of hustle culture on mental health and well-being are profound, and several studies have contributed to understanding its pervasive impact on students and young professionals.

\textcite{maharani2024prevalence} studied hustle culture among high school students, finding that excessive engagement in academic activities led to increased stress and compromised mental health. Their research highlighted how early exposure to productivity pressures shapes future workplace behavior.

\textcite{Tian2024} examined social media's impact on graduates, demonstrating how constant exposure to curated success stories creates unrealistic standards and contributes to overwork tendencies. Their findings suggest that social media significantly influences the development of hustle mentality among young professionals.

\textcite{Arora2021} analyzed employee motivation in India, focusing on the intersection of cultural factors and workplace dynamics. Their study revealed how traditional values and familial obligations influence work attitudes in Indian corporate settings.

